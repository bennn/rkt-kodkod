\section{Motivation}
\label{sec:motivation}

The \href{https://emina.github.io/rosette/}{Rosette} solver-aided language
 is an interesting research project aimed at spanning a gap between formal
 methods and programming.
Rosette's goal is to help programming language designers incorporate
 SMT and/or SAT solver technology in domain-specific programming languages (DSLs).
It does so by implementing a compiler from a high-level language (suitable
 for general purpose programming or for designing DSLs) to solver formulas.

Until recently, Rosette gave users the choice of three solver backends:
 Microsoft's \href{https://github.com/Z3Prover/z3/wiki}{Z3} solver,
 NYU's \href{http://cvc4.cs.nyu.edu/web/}{CVC4},
 and MIT's \href{http://alloy.mit.edu/kodkod/index.html}{Kodkod} solver.
Kodkod was the default backend and Kodkod is implemented in Java.
So new Rosette users had to first install Racket\footnote{Rosette is implemented in Racket.}
 and then install Java before they could begin using Rosette's solver-aided languages.

My goal was to move the Java implementation to Racket, with the hope that
 all of Rosette could be implemented in a single language.
As a secondary goal, I wanted to learn how to implement a model checker
 and at least ``go through the motions'' of writing one.\footnote{Hunter S. Thompson used to ``go through the motions'' of writing a great novel by copying pages from \emph{The Great Gatsby}~\cite{ny}. There are worse role models.}

Of these three goals, only 1 was really successful:
 I learned plenty about how Kodkod is implemented.
But I have much more to learn, because to date I have only written a compiler from
 propositional logic to CNF.
Hence the revised name of this semester project: Kodkod $\sqrt{2}$.
One checkpoint has been implemented, but the road to completing the solver
 stretched on far longer than I expected, and seemed to increase the more I
 learned.

The reason my second goal of implementing all Rosette in Racket has failed is
 due to a technicality.
Kodkod is designed to use a \emph{fast} external SAT solver and all the best
 are written in C.


